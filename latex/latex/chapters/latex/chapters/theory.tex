\section{Le modèle ThreeME standard}

\subsection{Structure générale}
Le modèle ThreeME est un modèle d'équilibre général calculable multi-sectoriel, caractérisé par :
\begin{itemize}
    \item Une structure néo-keynésienne
    \item Des rigidités nominales
    \item Une désagrégation sectorielle fine
    \item Une modélisation explicite de l'énergie
\end{itemize}

\subsection{Équations fondamentales}
La production sectorielle suit une fonction CES imbriquée :
\begin{equation}
Y_{j,t} = \left[\alpha_j(A_{K,t}K_{j,t})^{\frac{\sigma-1}{\sigma}} + 
(1-\alpha_j)(A_{L,t}L_{j,t})^{\frac{\sigma-1}{\sigma}}\right]^{\frac{\sigma}{\sigma-1}}
\end{equation}

où :
\begin{itemize}
    \item $Y_{j,t}$ est la production du secteur j
    \item $K_{j,t}$ est le stock de capital
    \item $L_{j,t}$ est le travail
    \item $\sigma$ est l'élasticité de substitution
    \item $\alpha_j$ est le paramètre de distribution
\end{itemize}

\section{Introduction des contraintes d'offre}

\subsection{Secteur agricole}
La production agricole est désormais contrainte par :

\begin{equation}
Y_{agr,t} = \min(Y_{pot,t}, D_t)
\end{equation}

où $Y_{pot,t}$ est la production potentielle affectée par les rendements.

\subsection{Ajustement des prix}
Le mécanisme d'ajustement des prix devient :

\begin{equation}
P_t = P_{t-1}\left(1 + \phi\frac{D_t - Y_{pot,t}}{Y_{pot,t}}\right)
\end{equation}

où $\phi$ est la vitesse d'ajustement.

\section{Intégration des dommages climatiques}

\subsection{Canaux de transmission}
Les dommages climatiques affectent l'économie via :
\begin{itemize}
    \item La productivité des facteurs
    \item Les rendements agricoles
    \item La production d'énergie
    \item Les événements extrêmes
    \item Le commerce international
\end{itemize}

\subsection{Spécification dynamique}
L'évolution du système est décrite par :

\begin{equation}
\begin{pmatrix}
Y_t \\ P_t \\ L_t \\ I_t
\end{pmatrix} = 
M(\Delta T)\begin{pmatrix}
Y_{t-1} \\ P_{t-1} \\ L_{t-1} \\ I_{t-1}
\end{pmatrix} +
\begin{pmatrix}
\epsilon^Y_t \\ \epsilon^P_t \\ \epsilon^L_t \\ \epsilon^I_t
\end{pmatrix}
\end{equation}
