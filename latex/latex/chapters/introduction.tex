\section{Contexte}
Les changements climatiques représentent un défi majeur pour l'économie mondiale. L'évaluation quantitative de leurs impacts est essentielle pour guider les politiques publiques.

\section{Objectifs du cours}
Ce cours vise à :
\begin{itemize}
    \item Comprendre les mécanismes de transmission des dommages climatiques
    \item Maîtriser la modélisation avec le modèle ThreeME modifié
    \item Savoir estimer les fonctions de dommages
    \item Pouvoir analyser différents scénarios climatiques
\end{itemize}

\section{Structure du cours}
Le cours est organisé en quatre parties principales :
\begin{enumerate}
    \item Fondements théoriques du modèle ThreeME
    \item Estimation des fonctions de dommages sectoriels
    \item Implémentation pratique du modèle
    \item Analyse des résultats et implications politiques
\end{enumerate}

\section{Prérequis}
Pour suivre ce cours, il est nécessaire d'avoir :
\begin{itemize}
    \item Des bases solides en macroéconomie
    \item Une connaissance du langage R
    \item Des notions de modélisation économique
    \item Une compréhension des enjeux climatiques
\end{itemize}
